% !TeX program = XeLaTeX

\documentclass[12pt,twoside]{article}
\usepackage[margin=0.75in]{geometry}
\usepackage{multicol}
\usepackage{dblfnote}
\usepackage{lipsum}
\usepackage{fancyhdr}
\usepackage{lmodern}
\usepackage{lmodern}
\usepackage{graphicx}
\usepackage{changepage}
\usepackage{longtable}
\usepackage[dvipsnames]{xcolor}
\usepackage[hebrew, english]{babel}
\selectlanguage{english}


\setlength{\columnsep}{1cm}

\interfootnotelinepenalty=10000

\makeatletter
\renewcommand{\@makefntext}[1]{%
  \noindent#1\par}  % Just the text, no indent, no label, no dash
\makeatother

% Superscript verse number
\newcommand{\vs}[1]{\textsuperscript{#1}}

% Custom study note footnote, markerless, with custom label inside
\newcommand{\vnote}[2]{%
  \begingroup
  \renewcommand\thefootnote{}%
  \footnote{\scriptsize \textbf{}#2}%
  \addtocounter{footnote}{-1}%
  \endgroup
}

\renewcommand{\footnoterule}{%
  \vspace{0.3em} % Space above the line
  \noindent\textcolor{MidnightBlue}{\noindent\rule{\textwidth}{2pt}} % The line itself
  \vspace{-0.1em} % Space below the line
}

\newcommand{\chapterWithIndent}[2]{%
  \noindent
  \begin{minipage}[t]{1cm}
    \vspace{-0.4\baselineskip}
    {\textcolor{MidnightBlue}{\fontsize{40pt}{48pt}\selectfont \textbf{#1}}}
  \end{minipage}%
  \hspace{0.3cm}%
  \begin{minipage}[t]{\dimexpr\linewidth - 1.5cm - 0.3cm\relax}
    \hangindent=1.5em
    \hangafter=3
    #2
    \vspace{0.05cm}
  \end{minipage}
}

\newcommand{\chapterWithBigIndent}[2]{%
  \noindent
  \begin{minipage}[t]{1cm}
    \vspace{-0.4\baselineskip}
    {\textcolor{MidnightBlue}{\fontsize{40pt}{48pt}\selectfont \textbf{#1}}}
  \end{minipage}%
  \hspace{0.9cm}%
  \begin{minipage}[t]{\dimexpr\linewidth - 1.5cm - 0.3cm\relax}
    \hangindent=1.5em
    \hangafter=3
    #2
    \vspace{0.05cm}
  \end{minipage}
}


\setlength{\columnsep}{0.5cm}
\setlength{\headheight}{14.49998pt}

\pagestyle{fancy}
\fancyhf{} % Clear all header and footer fields
\fancyhead[LO,RE]{\thepage}% Left on odd pages, right on even pages

\begin{document}

\begin{center}
{\textcolor{MidnightBlue}{Introduction to the Epistle to the}}\\
\vspace{0.5cm}
{\textcolor{MidnightBlue}{\fontsize{40pt}{48pt}\selectfont \textbf{GALATIANS}}}\\
\vspace{2cm}
\vspace{-1cm}
\end{center}
\thispagestyle{fancy}

\newpage


%\begin{multicols}{2}

\selectlanguage{english}
\chapterWithIndent{1}{
    \vs{1} Paul, an apostle---sent not from men nor by a man, but by Jesus Christ and God the Father, who raised him from the dead---
    \vs{2} and all the brothers and sisters with me,\vspace{0.3cm}
}

To the churches in Galatia:\vspace{0.3cm}\vnote{1}{\textbf{1:1} - Paul makes it clear that his authority as an apostle is \textcolor{MidnightBlue}{divine} and not institutional. This foreshadows his upcoming argument: that he is not a secondhand messenger, but one commissioned directly by God. Look at verses 11-12 for more emphasis on this argument.}
\vnote{2}{\textbf{1:2} - \textcolor{MidnightBlue}{Galatia} was a Roman province in Asia Minor who were deeply conservative in their Jewish tradition, particularly the school of \textcolor{MidnightBlue}{Shammei} (See Intertestamonial Period Notes). It was a haven for Jews seeking to avoid Greco-Roman influence and maintain strict adherence to the Law of Moses (See Galatians Intro for more).}

\vs{3} Grace and peace to you from God our Father and the Lord Jesus Christ,\vnote{3}{\textbf{1:3} - \textcolor{MidnightBlue}{"Grace and Peace"}: A fusion of the Greek \textit{charis} and Hebrew \textit{shalom}, which were greetings in their respective cultures. This shows Jew + Gentile unity even in the greeting.}
\vs{4} who gave himself for our sins to rescue us from the present evil age, according to the will of our God and Father,\vnote{4}{\textbf{1:4} - The rescue is from the \textcolor{MidnightBlue}{"present evil age"}, echoing apocalyptic Jewish expectation - not just personal sin, but system oppression.}
\vs{5} to whom be glory for ever and ever. Amen.

\subsection*{\textcolor{MidnightBlue}{\textbf{No Other Gospel}}}

\hspace{0.5cm} \vs{6} I am astonished that you are so quickly deserting the one who called you to live in the grace of Christ and are turning to a different gospel---\vnote{6}{\textbf{1:6} - The problem here is that God-fearing Gentiles (\textcolor{MidnightBlue}{theosebes}) were being pressured to convert to Judaism to gain full inclusion. This means adhering to Jewish law and customs such as circumcision, dietary laws, etc. Otherwise, theosebes living in this region were relegated to the margins of the Jewish community. In addition, the Jewish community was granted legal protection under the Roman Empire, which meant Jews in this region were not required to worship the emperor. This was known as the \textcolor{MidnightBlue}{Jewish Exception}.}

\vnote{6}{Gentiles who were not included were forced between two choices: convert to Judaism and gain legal protection, or be forced to worship the emperor and commit idolatry. This is so many theosebes were \textcolor{MidnightBlue}{"so quickly deserting"} the gospel that has been made known to them by Paul. This 'different gospel' likely refers to Torah conversion, not paganism, implicating that inclusion by conversion is a distortion of the gospel.}
\vs{7} which is really no gospel at all. Evidently some people are throwing you into confusion and are trying to pervert the gospel of Christ.
\vs{8} But even if we or an angel from heaven should preach a gospel other than the one we preached to you, let them be under God's curse!\vnote{8}{\textbf{1:8, 9} - Very strong language being used here. The Greek, \textit{anathema}, means excommunicated, or cut off, from the covenant community.}
\vs{9} As we have already said, so now I say again: If anybody is preaching to you a gospel other than what you accepted, let them be under God's curse!

\vs{10} Am I now trying to win the approval of human beings, or of God? Or am I trying to please people? If I were still trying to please people, I would not be a servant of Christ.

\subsection*{\textcolor{MidnightBlue}{\textbf{Paul Called by God}}}

\hspace{0.5cm} \vs{11} I want you to know, brothers and sisters, that the gospel I preached is not of human origin.
\vs{12} I did not receive it from any man, nor was I taught it; rather, I received it by revelation from Jesus Christ.

\vs{13} For you have heard of my previous way of life in Judaism, how intensely I persecuted the church of God and tried to destroy it.
\vs{14} I was advancing in Judaism beyond many of my own age among my people and was extremely zealous for the traditions of my fathers.\vnote{14}{\textbf{1:13-14} - Paul highlights his former zeal for Judaism. As the student of Gamaliel, a leading Pharisee, Paul had much status and credibility in the Jewish community. The \textcolor{MidnightBlue}{"traditions of my fathers"} implies rabbinic interpretations (likely Shammei's camp), and not just the Torah itself. This is important because it shows that Paul's conversion was not just a change of belief, but a radical transformation of his entire worldview. His calling did not make his life any easier. He had to give up social status and rabbinic prestige, and even personal safety. He is showing this Gentile audience that following Jesus' gospel is costly - both for Paul and Gentile theosebes refusing to bow to Caesar or convert to Torah adherence for convenience.}
\vs{15} But when God, who set me apart from my mother's womb and called me by his grace, was pleased\vnote{15}{\textbf{1:15} - Paul parallels Old Testament prophets like Jeremiah (Je 1:5) and Isaiah (Is 49:1).}
\vs{16} to reveal his Son in me so that I might preach him among the Gentiles, my immediate response was not to consult any human being.
\vs{17} I did not go up to Jerusalem to see those who were apostles before I was, but I went into Arabia. Later I returned to Damascus.\vnote{17}{\textbf{1:17} - Paul immediately withdrew into solitude in \textcolor{MidnightBlue}{Arabia} and then \textcolor{MidnightBlue}{Damascus} after his conversion. This signifies a season of training under the yoke of Christ. It may also symbolize wilderness transformations like that of Moses (Ex 3:1-4:17) and Elijah (1 Ki 19:1-18), or even Jesus himself (Mt 4:1-11). This is a time of preparation before his public ministry.}

\vs{18} Then after three years, I went up to Jerusalem to get acquainted with Cephas and stayed with him fifteen days.\vnote{18}{\textbf{1:18} - The \textcolor{MidnightBlue}{three years} is symbolic of rabbinic discipleship taking 3+ years. Paul simply checks his gospel with the leaders of the faith in Jerusalem, but not to be taught by them.}
\vs{19} I saw none of the other apostles---only James, the Lord's brother.\vnote{19}{\textbf{1:19} - James here is referred to as \textcolor{MidnightBlue}{"the Lord's brother"}, though this could mean cousin, relative, or tight kinship group, not necessarily in a biological sense.}
\vs{20} I assure you before God that what I am writing you is no lie.

\vs{21} Then I went to Syria and Cilicia.
\vs{22} I was personally unknown to the churches of Judea that are in Christ.\vnote{22}{\textbf{1:22} - Paul remained relatively obscure in Judea. His ministry started on the fringes.}
\vs{23} They only heard the report: ``The man who formerly persecuted us is now preaching the faith he once tried to destroy.''
\vs{24} And they praised God because of me.\vnote{24}{\textbf{1:23-24} - His radical transformation caused others to glorify God. This proves the fruit of the gospel: God is at work in the Gentile world through Paul.}
\vnote{24}{\textbf{Ch 1 Discussion} - The issue at play from chapter 1 is not how humans earn salvation. It is specifically how Gentiles can gain full membership in God's covenant family. The core of Paul's gospel, as some have called the Epistle to the Galatians, isn't a new doctrine of substitutionary atonement, but the \textcolor{MidnightBlue}{radical inclusion of the Gentiles} in God's family without requiring Torah conversion. Faith in Jesus is sufficient for belonging. Ask yourself these questions:
\begin{itemize}
    \item How do we define "another gospel" today? What "good news" is really \textit{not} good news?
    \item What systems today subtly suggest some people need to "convert" to belong?
    \item What is the risk (and reward) of living out a Gospel of inclusion?
\end{itemize}
}

\subsection*{{\textcolor{MidnightBlue}{\textbf{Paul Accepted by the Apostles}}}}

\chapterWithIndent{2}{
    \vs{1} Then after fourteen years, I went up again to Jerusalem, this time with Barnabas. I took Titus along also.
    \vs{2} I went in response to a revelation and, meeting privately with those esteemed as leaders, I presented to them the gospel that I preach among the Gentiles. I
}

\noindent wanted to be sure I was not running and had not been running my race in vain.
\vs{3} Yet not even Titus, who was with me, was compelled to be circumcised, even though he was a Greek.\vnote{1}{\textbf{2:1} - Paul visits the leaders of this early church \textcolor{MidnightBlue}{fourteen years} after his three-year stint in Arabia/Damascus to reaffirm he is on the right path.}\vnote{1}{\textbf{2:1-3} - \textcolor{MidnightBlue}{Titus} was used here almost as a test case for Paul's point. He was a Cretan, i.e. the epitome of a pagan Gentile (known for being barbaric), and if he was to be accepted as-is (without circumcision), then all Gentiles are. \textcolor{MidnightBlue}{Circumcision} was more than a medical or religious act. It was a symbolic identity marker of full covenantal Jewish belonging, so by asking if Titus should be circumcised, Paul is really asking if Gentiles should become Jewish (keep kosher, wear tzitzit, etc.) to be included in the people of God.}
\vs{4} This matter arose because some false believers had infiltrated our ranks to spy on the freedom we have in Christ Jesus and to make us slaves.
\vs{5} We did not give in to them for a moment, so that the truth of the gospel might be preserved for you.

\vs{6} As for those who were held in high esteem---whatever they were makes no difference to me; God does not show favoritism---they added nothing to my message.
\vs{7} On the contrary, they recognized that I had been entrusted with the task of preaching the gospel to the uncircumcised, just as Peter had been to the circumcised.
\vs{8} For God, who was at work in Peter as an apostle to the circumcised, was also at work in me as an apostle to the Gentiles.
\vs{9} James, Cephas, and John, those esteemed as pillars, gave me and Barnabas the right hand of the fellowship when they recognized the grace given to me. They agreed that we should go to the Gentiles, and they to the circumcised.\vnote{9}{\textbf{2:9} - Peter, James and John, the leaders of the early church seemed to add \textcolor{MidnightBlue}{nothing} to Paul's Gospel message, but instead extended the \textcolor{MidnightBlue}{"right hand of fellowship"}, a symbolic Roman handshake (forearm-to-forearm), signifying brotherhood, mutual acceptance and shared identity.}\vnote{9}{\textbf{2:9} - The three apostles are referred to here as \textcolor{MidnightBlue}{"pillars"}, which in Jewish thought refers to Abraham, Isaac and Jacob. Paul cleverly aligns Peter, James and John with this lineage. Some scholars even think he is giving a nod to himself as the \textcolor{MidnightBlue}{fourth pillar}, which parallels the "fourth patriarch", Joseph, as this was the extra patriarch in the Torah narrative. Rabbinically, these three church leaders, the ones closest to their Rabbi, Jesus, would have been the only real options to head this new church movement.}
\vs{10} All they asked was that we should continue to remember the poor, the very thing I had been eager to do all along.

\subsection*{\textcolor{MidnightBlue}{\textbf{Paul Opposes Cephas}}}

\hspace{0.5cm} \vs{11} When Cephas came to Antioch, I opposed him to his face, because he stood condemned.\vnote{11}{\textbf{2:11-13} - Peter (Cephas here) ate with Gentiles until \textcolor{MidnightBlue}{"men from James"}, or Judean halakhic traditionalists, arrived. Peter's "drawing back", clearly to avoid criticism, caused others to do the same. He knew that the Gospel included Gentiles, yet reverted to old customs under pressure. It is reassuring to know that even the pillars of the church were human and reminds us that we are all full of mistakes.}\vnote{11}{\textcolor{MidnightBlue}{Halakhah}, or "the way one walks", refers to the oral traditions and rabbinic rulings that guide how one keeps written Torah. Think of it as an interpretive fence surrounding Torah to further their walk. This is a perfect example of \textit{halakhah}, as Torah never forbids eating with Gentiles, but this tradition did, for purity and distinctiveness. These Jewish disciples of Jesus clearly have a hard time shaking off their \textit{halakhah}. This is something they had lived with all of their lives, and when these Judeans came into town, they resorted back to it to please them.}
\vs{12} For before certain men came from James, he used to eat with the Gentiles. But when they arrived, he began to draw back and separate himself from the Gentiles because he was afraid of those who belonged to the circumcision group.
\vs{13} The other Jews joined him in his hypocrisy, so that by their hypocrisy even Barnabas was led astray.

\vs{14} When I saw that they were not acting in line with the truth of the gospel, I said to Cephas in front of them all, ``You are a Jew, yet you live like a Gentile and not like a Jew. How is it, then, that you force Gentiles to follow Jewish customs?''

\vs{15} ``We who are Jews by birth and not sinful Gentiles
\vs{16} know that a person is not justified by the works of the law, but by faith in Jesus Christ. So we, too, have put out faith in Christ Jesus that we may be justified by faith in Christ and not by the works of the law, because by the works of the law no one will be justified.\vnote{16}{\textbf{2:15-16} - Paul explains here that Jewish customs (Torah and \textit{halakhah}) are not the source of justification. \textcolor{MidnightBlue}{Justification} is NOT the same as salvation, but being declared righteous by God. Despite the common belief, no Jew has ever believed that you earn salvation by works, but that it was by grace. The debate was why God declares some righteous: Shammei believed in justification through the works of the law (\textit{Miqsat Ma'ase haTorah}), and Hillel believed in justification through faith, like Abraham. Paul clearly sides with Hillel.}

\vs{17} ``But if, in seeking to be justified in Christ, we Jews find ourselves also among the sinners, doesn't that mean that Christ promotes sin? Absolutely not!
\vs{18} If I rebuild what I destroyed, then I really would be a lawbreaker.

\vs{19} ``For through the law I died to the law so that I might live for God.
\vs{20} I have been crucified with Christ and I no longer live, but Christ lives in me. The life I now live in the body, I live by faith in the Son of God, who loved me and gave himself for me.\vnote{20}{\textbf{2:20} - The Gospel is about \textcolor{MidnightBlue}{transformation} of the heart, not Torah-based identity.}
\vs{21} I do not set aside the grace of God, for if righteousness could be gained through the law, Christ died for nothing!''\vnote{21}{\textbf{2:21} - Paul is not anti-law, and he still lives very Jewishly (eats kosher, observes feasts, etc.) but Torah taught him its own limits (see vs. 19), but obedience to it cannot justify, or declare someone righteous.}
\vnote{21}{\textbf{Ch 2 Discussion} - Paul shows in this chapter that he values community, respects authority, and seeks accountability. It is to note that even the great Paul seeks blessing and affirmation from the three pillars of the church. He is even willing to end his mission completely if his gospel didn't align with theirs', as his race would then have been run in vain (vs. 2). On the contrary, he stands firm against Peter when the Gospel is threatened - he desires to promote truth, even if it demands confrontation with one who was ``held in high esteem". This early church can be a model of grappling faithfully with disagreements, instead of silencing them completely. Ask yourself these questions:
\begin{itemize}
    \item When have you seen peer pressure or tradition get in the way of the true Gospel message?
    \item How do you balance respect for authority with standing firm in what you believe?
    \item What does it mean in your life to say, ``I have been crucified with Christ"?
\end{itemize}
}

\subsection*{\textcolor{MidnightBlue}{\textbf{Faith or Works of the Law}}}

\chapterWithIndent{3}{
    \vs{1} You foolish Galatians! Who has bewitched you? Before your very eyes Jesus Christ was clearly portrayed as crucified.
    \vs{2} I would like to learn just one thing from you: Did you receive the Spirit by the works of the law, or by believing what you heard?
    \vs{3} Are you
}

 \noindent so foolish? After beginning by means of the Spirit, are you now trying to finish by means of the flesh?\vnote{1}{\textbf{3:1} - Based on Paul's harsh and angry language here, it seems as if the Galatians should have already known what Paul had previously mentioned. The very phrase, \textcolor{MidnightBlue}{"foolish Galatians"}, was a cultural play. Galatia was a backwater region, and considered more of an uncivilized and barbaric region of the Roman Empire. These Shammei Jews who moved here however were a very educated group, which made for the perfect jab.}\vnote{2}{\textbf{3:2} - Paul is addressing the \textit{theosebes}, so of course the answer to his question would be that they received the Spirit by \textcolor{MidnightBlue}{faith} in what they heard. It is a  rhetorical question. Faith is there entire story, so why would they want to make it about the works of the law?}
 \vs{4} Have you experienced so much in vain---if it really was in vain?\vnote{4}{\textbf{3:4} - This verse implies there has been a level of \textcolor{MidnightBlue}{suffering}. They've suffered persecution from both Jews for being too lax, and Romans for being too Jewish (not included under Jewish Exception) - why throw it all away now by converting? Their Gospel will be put on display in the suffering that they have experienced.}
 \vs{5} So again I ask, does God give you his Spirit and work miracles among you by the works of the law, or by your believing what you heard?
 \vs{6} So also Abraham ``believed God, and it was credited to him as righteousness.''\vnote{6}{\textbf{3:6} - This is the rabbinic argument used by Hillel to prove that Abraham was justified by faith. Paul (and Hillel) reference Ge 15:6, which states that Abraham is made righteous two chapters before God commanded him to be circumcised (Ge 17)! Note that God had given Abraham commands (leave his land/family, cut animals, etc.), but circumcision stood as the first command in a \textcolor{MidnightBlue}{covenantal} relationship.}

 \vs{7} Understand, then, that those who have faith are children of Abraham.\vnote{7}{\textbf{3:7-9} - Paul takes the argument one step further than Hillel, and says that the Gospel was first revealed to Abraham, and that \textit{theosebes} with faith are adopted as true \textcolor{MidnightBlue}{children of Abraham---\textit{b'nei Avraham}}. Ge 12:3, as quoted by Paul, is what he calls the Gospel. The Gospel message has always been about trusting the story of God and not obedience to the law.}
 \vs{8} Scripture foresaw that God would justify the Gentiles by faith, and announced the gospel in advance to Abraham: ``All nations will be blessed through you.''
 \vs{9} So those who rely on faith are blessed along with Abraham, the man of faith.

 \vs{10} For all who rely on the works of the law are under a curse, as it is written: ``Cursed is everyone who does not continue to do everything written in the Book of the Law.''\vnote{10}{\textbf{3:10} - De 27:26 is quoted, and it's curse language from Mount Ebal/Gerizim is used in this way: if you commit to the \textit{Miqsat Ma'ase haTorah} as your justification, you must keep \textit{all} of it. This is not a curse \textit{from God}, but the \textcolor{MidnightBlue}{psychological and spiritual burden} of legalism. The curse is the sheer weight of trying to live a perfect life.}
 \vs{11} Clearly no one who relies on the law is justified before God, because ``the righteous will live by faith.''\vnote{11}{\textbf{3:11-12} - Paul uses a brilliant tool called a \textcolor{MidnightBlue}{\textit{Gezerah Shavah}}, where he takes the same phrase located in two different passages, and ties those passages together. Here, he (preceded by Hillel, who did the same thing) notes in vs. 11 that the prophet Habakkuk (2:4) talked about living by \textcolor{MidnightBlue}{\textit{emunah}}, when he is talking about righteousness. This word means faithfulness, or putting your faith into action, not just faith. He connects this idea with a quote from Leviticus (18:5) in vs. 12. Paul is making the case that the law is not there for our justification, but to teach us how to live with \textit{emunah}.}
 \vs{12} The law is not based on faith; on the contrary, it says, ``The person who does these things will live by them.''
 \vs{13} Christ redeemed us from the curse of the law by becoming a curse for us, for it is written: ``Cursed is everyone who is hung on a pole.''
 \vs{14} He redeemed us in order that the blessing given to Abraham might come to the Gentiles through Christ Jesus, so that by faith we might receive the promise of the Spirit.\vnote{14}{textbf{3:13-14} - Paul follows up one \textit{gezerah shavah} with another. He connects the curse to Jesus, by referring to Christ's crucifixion in reference to being \textcolor{MidnightBlue}{``hung on a pole"} (De 21:23). Paul claims that when Jesus subjected himself to crucifixion, he took the weight of that curse and redeemed us from that faulty way of thinking.}

 \subsection*{\textcolor{MidnightBlue}{\textbf{The Law and the Promise}}}

 \hspace{0.5cm} \vs{15} Brothers and sisters, let me take an example from everyday life. Just as no one can set aside or add to a human covenant that has been duly established, so it is in this case.
 \vs{16} The promises were spoken to Abraham and to his seed. Scripture does not say ``and to seeds,'' meaning many people, but ``and to your seed,'' meaning one person, who is Christ.
 \vs{17} What I mean is this: The law, introduced 430 years later, does not set aside the covenant previously established by God and thus do away with the promise.
 \vs{18} For if the inheritance depends on the law, then it no longer depends on the promise; but God in his grace gave it to Abraham through a promise.

 \vs{19} Why, then, was the law given at all? It was added because of the transgressions until the Seed to whom the promise referred had come. The law was given through angels and entrusted to a mediator.
 \vs{20} A mediator, however, implies more than one party; but God is one.

 \vs{21} Is the law, therefore, opposed to the promises of God? Absolutely not! For if a law had been given that could impart life, then righteousness would certainly have come by the law.
 \vs{22} But Scripture has locked up everything under the control of sin, so that what was promised, being given through faith in Jesus Christ, might be given to those who believe.

 \subsection*{\textcolor{MidnightBlue}{\textbf{Children of God}}}

 \hspace{0.5cm} \vs{23} Before the coming of this faith, we were held in custody of under the law, locked up until the faith that was to come would be revealed.
 \vs{24} So the law was our guardian until Christ came that we might be justified by faith.
 \vs{25} Now that this faith has come, we are no longer under a guardian.

 \vs{26} So in Christ Jesus you are all children of God through faith,
 \vs{27} for all of you who were baptized into Christ have clothed yourselves with Christ.
 \vs{28} There is neither Jew nor Gentile, neither slave nor free, nor is there male and female, for you are all one in Christ Jesus.
 \vs{29} If you belong to Christ, then you are Abraham's seed, and heirs according to the promise.


\end{document}
