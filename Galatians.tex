% !TeX program = XeLaTeX

\documentclass[12pt,twoside]{article}
\usepackage[margin=0.75in]{geometry}
\usepackage{multicol}
\usepackage{dblfnote}
\usepackage{lipsum}
\usepackage{fancyhdr}
\usepackage{lmodern}
\usepackage{lmodern}
\usepackage{graphicx}
\usepackage{changepage}
\usepackage{longtable}
\usepackage[dvipsnames]{xcolor}
\usepackage[hebrew, english]{babel}
\selectlanguage{english}


\setlength{\columnsep}{1cm}

\interfootnotelinepenalty=10000

\makeatletter
\renewcommand{\@makefntext}[1]{%
  \noindent#1\par}  % Just the text, no indent, no label, no dash
\makeatother

% Superscript verse number
\newcommand{\vs}[1]{\textsuperscript{#1}}

% Custom study note footnote, markerless, with custom label inside
\newcommand{\vnote}[2]{%
  \begingroup
  \renewcommand\thefootnote{}%
  \footnote{\scriptsize \textbf{}#2}%
  \addtocounter{footnote}{-1}%
  \endgroup
}

\renewcommand{\footnoterule}{%
  \vspace{0.3em} % Space above the line
  \noindent\textcolor{MidnightBlue}{\noindent\rule{\textwidth}{2pt}} % The line itself
  \vspace{-0.1em} % Space below the line
}

\newcommand{\chapterWithIndent}[2]{%
  \noindent
  \begin{minipage}[t]{1cm}
    \vspace{-0.4\baselineskip}
    {\textcolor{MidnightBlue}{\fontsize{40pt}{48pt}\selectfont \textbf{#1}}}
  \end{minipage}%
  \hspace{0.3cm}%
  \begin{minipage}[t]{\dimexpr\linewidth - 1.5cm - 0.3cm\relax}
    \hangindent=1.5em
    \hangafter=3
    #2
    \vspace{0.05cm}
  \end{minipage}
}

\newcommand{\chapterWithBigIndent}[2]{%
  \noindent
  \begin{minipage}[t]{1cm}
    \vspace{-0.4\baselineskip}
    {\textcolor{MidnightBlue}{\fontsize{40pt}{48pt}\selectfont \textbf{#1}}}
  \end{minipage}%
  \hspace{0.9cm}%
  \begin{minipage}[t]{\dimexpr\linewidth - 1.5cm - 0.3cm\relax}
    \hangindent=1.5em
    \hangafter=3
    #2
    \vspace{0.05cm}
  \end{minipage}
}


\setlength{\columnsep}{0.5cm}
\setlength{\headheight}{14.49998pt}

\pagestyle{fancy}
\fancyhf{} % Clear all header and footer fields
\fancyhead[LO,RE]{\thepage}% Left on odd pages, right on even pages

\begin{document}

\begin{center}
{\textcolor{MidnightBlue}{\fontsize{40pt}{48pt}\selectfont \textbf{Galatians}}}\\
\vspace{2cm}
\vspace{-1cm}
\end{center}
\thispagestyle{fancy}


%\begin{multicols}{2}

\selectlanguage{english}
\chapterWithIndent{1}{
    \vs{1} Paul, an apostle---sent not from men nor by a man, but by Jesus Christ and God the Father, who raised him from the dead---
    \vs{2} and all the brothers and sisters with me,\vspace{0.5cm}
}

To the churches in Galatia:\vspace{0.5cm}\vnote{1}{\textbf{1:1} - Paul makes it clear that his authority as an apostle is \textit{divine} and not institutional. This foreshadows his upcoming argument: that he is not a secondhand messenger, but one commissioned directly by God. Look at verses 11-12 for more emphasis on this argument.}
\vnote{2}{\textbf{1:2} - Galatia was a Roman province in Asia Minor who were deeply conservative in their Jewish tradition, particularly the school of \textit{Shammei} (See Intertestamonial Period Notes). It was a haven for Jews seeking to avoid Greco-Roman influence and maintain strict adherence to the Law of Moses (See Galatians Intro for more).}

\vs{3} Grace and peace to you from God our Father and the Lord Jesus Christ,\vnote{3}{\textbf{1:3} - 'Grace and Peace': A fusion of the Greek \textit{charis} and Hebrew \textit{shalom}, which were greetings in their respective cultures. This shows Jew + Gentile unity even in the greeting.}
\vs{4} who gave himself for our sins to rescue us from the present evil age, according to the will of our God and Father,\vnote{4}{\textbf{1:4} - The rescue is from the 'present evil age', echoeing apocalyptic Jewish expectation - not just personal sin, but system oppression.}
\vs{5} to whom be glory for ever and ever. Amen.

\subsection*{\textcolor{MidnightBlue}{\textbf{No Other Gospel}}}

\hspace{0.5cm} \vs{6} I am astonished that you are so quickly deserting the one who called you to live in the grace of Christ and are turning to a different gospel---\vnote{6}{\textbf{1:6} - The problem here is that God-fearing Gentiles (theosebes) were being pressured to convert to Judaism to gain full inclusion. This means adhering to Jewish law and customs such as circumcision, dietary laws, etc. Otherwise, theosebes living in this region were relegated to the margins of the Jewish community. In addition, the Jewish community was granted legal protection under the Roman Empire, which meant Jews in this region were not required to worship the emperor. This was known as the Jewish Exception. Gentiles who were not included were forced between two choices: convert to Judaism and gain legal protection, or be forced to worship the emperor and commit idolatry. This is so many theosebes were 'so quickly deserting' the gospel that has been made known to them by Paul. This 'different gospel' likely refers to Torah conversion, not paganism, implicating that inclusion by conversion is a distortion of the gospel.}
\vs{7} which is really no gospel at all. Evidently some people are throwing you into confusion and are trying to pervert the gospel of Christ.
\vs{8} But even if we or an angel from heaven should preach a gospel other than the one we preached to you, let them be under God's curse!\vnote{8}{\textbf{1:8, 9} - Very strong language being used here. The Greek, \textit{anathema}, means excommunicated, or cut off, from the covenant community.}
\vs{9} As we have already said, so now I say again: If anybody is preaching to you a gospel other than what you accepted, let them be under God's curse!

\vs{10} Am I now trying to win the approval of human beings, or of God? Or am I trying to please people? If I were still trying to please people, I would not be a servant of Christ.

\subsection*{\textcolor{MidnightBlue}{\textbf{Paul Called by God}}}

\hspace{0.5cm} \vs{11} I want you to know, brothers and sisters, that the gospel I preached is not of human origin.
\vs{12} I did not receive it from any man, nor was I taught it; rather, I received it by revelation from Jesus Christ.

\vs{13} For you have heard of my previous way of life in Judaism, how intensely I persecuted the church of God and tried to destroy it.
\vs{14} I was advancing in Judaism beyond many of my own age among my people and was extremely zealous for the traditions of my fathers.\vnote{14}{\textbf{1:13-14} - Paul highlights his former zeal for Judaism. As the student of Gamaliel, a leading Pharisee, Paul had much status and credibility in the Jewish community. The 'traditions of my fathers' implies rabbinic interpretations (likely Shammei's camp), and not just the Torah itself. This is important because it shows that Paul's conversion was not just a change of belief, but a radical transformation of his entire worldview. His calling did not make his life any easier. He had to give up social status and rabbinic prestige, and even personal safety. He is showing this Gentile audience that following Jesus' gospel is costly - both for Paul and Gentile theosebes refusing to bow to Caesar or convert to Torah adherence for convenience.}
\vs{15} But when God, who set me apart from my mother's womb and called me by his grace, was pleased\vnote{15}{\textbf{1:15} - Paul parallels Old Testament prophets like Jeremiah (Je 1:5) and Isaiah (Is 49:1).}
\vs{16} to reveal his Son in me so that I might preach him among the Gentiles, my immediate response was not to consult any human being.
\vs{17} I did not go up to Jerusalem to see those who were apostles before I was, but I went into Arabia. Later I returned to Damascus.\vnote{17}{\textbf{1:17} - Paul immediately withdrew into solitude in Arabia and then Damascus after his conversion. This signifies a season of training under the yoke of Christ. It may also symbolize wilderness transformations like that of Moses (Ex 3:1-4:17) and Elijah (1 Ki 19:1-18), or even Jesus himself (Mt 4:1-11). This is a time of preparation before his public ministry.}

\vs{18} Then after three years, I went up to Jerusalem to get acquainted with Cephas and stayed with him fifteen days.\vnote{18}{\textbf{1:18} - The 'three years' is symbolic of rabbinic discipleship taking 3+ years. Paul simply checks his gospel with the leaders of the faith in Jerusalem, but not to be taught by them.}
\vs{19} I saw none of the other apostles---only James, the Lord's brother.\vnote{19}{\textbf{1:19} - James here is referred to as 'the Lord's brother', though this could mean cousin, relative, or tight kinship group, not necessarily in a biological sense.}
\vs{20} I assure you before God that what I am writing you is no lie.

\vs{21} Then I went to Syria and Cilicia.
\vs{22} I was personally unknown to the churches of Judea that are in Christ.\vnote{22}{\textbf{1:22} - Paul remained relatively obscure in Judea. His ministry started on the fringes.}
\vs{23} They only heard the report: ``The man who formerly persecuted us is now preaching the faith he once tried to destroy.''
\vs{24} And they praised God because of me.\vnote{24}{\textbf{1:23-24} - His radical transformation caused others to glorify God. This proves the fruit of the gospel: God is at work in the Gentile world through Paul.}
\vnote{24}{\textbf{Ch 1 Discussion} - The issue at play from chapter 1 is not how humans earn salvation. It is specifically how Gentiles can gain full membership in God's covenant family. The core of Paul's gosepl, as some have called the Epistle to the Galatians, isn't a new doctrine of substitutionary atonement, but the radical inclusion of the Gentiles in God's family without requiring Torah conversion. Faith in Jesus is sufficient for belonging. Ask youself these questions:
\begin{itemize}
    \item How do we define "another gospel" today? What "good news" is really \textit{not} good news?
    \item What systems today subtly suggest some people need to "convert" to belong?
    \item What is the risk (and reward) of living out a Gospel of inclusion?
\end{itemize}
}

\subsection*{{\textcolor{MidnightBlue}{\textbf{Paul Accepted by the Apostles}}}}

\chapterWithIndent{2}{
    \vs{1} Then after fourteen years, I went up again to Jerusalem, this time with Barnabas. I took Titus along also.
    \vs{2} I went in response to a revelation and, meeting privately with those esteemed as leaders, I presented to them the gospel that I preach among the Gentiles. I
}

\noindent wanted to be sure I was not running and had not been running my race in vain.
\vs{3} Yet not even Titus, who was with me, was compelled to be circumcised, even though he was a Greek.
\vs{4} This matter arose because some false believers had infiltrated our ranks to spy on the freedom we have in Christ Jesus and to make us slaves.
\vs{5} We did not give in to them for a moment, so that the truth of the gospel might be preserved for you.

\vs{6} As for those who were held in high esteem---whatever they were makes no difference to me; God does not show favoritism---they added nothing to my message.
\vs{7} On the contrary, they recognized that I had been entrusted with the task of preaching the gospel to the uncircumcised, just as Peter had been to the circumcised.
\vs{8} For God, who was at work in Peter as an apostle to the circumcised, was also at work in me as an apostle to the Gentiles.
\vs{9} James, Cephas, and John, those esteemed as pillars, gave me and Barnabas the right hand of the fellowship when they recognized the grace given to me. They agreed that we should go to the Gentiles, and they to the circumcised.
\vs{10} All they asked was that we should continue to remember the poor, the very thing I had been eager to do all along.

\subsection*{\textcolor{MidnightBlue}{\textbf{Paul Opposes Cephas}}}

\hspace{0.5cm} \vs{11} When Cephas came to Antioch, I opposed him to his face, because he stood condemned.
\vs{12} For before certain men came from James, he used to eat with the Gentiles. But when they arrived, he began to draw back and separate himself from the Gentiles because he was afraid of thos who belonged to the circumcision group.
\vs{13} The other Jews joined him in his hypocricy, so that by their hypocricy even Barnabas was led astray.

\vs{14} When I saw that they were not acting in line with the truth of the gospel, I said to Cephas in front of them all, ``You are a Jew, yet you live like a Gentile and not like a Jew. How is it, then, that you force Gentiles to follow Jewish customs?''

\vs{15} ``We who are Jews by birth and not sinful Gentiles
\vs{16} know that a person is not justified by the works of the law, but by faith in Jesus Christ. So we, too, have put out faith in Christ Jesus that we may be justified by faith in Christ and not by the works of the law, because by the works of the law no one will be justified.

\vs{17} ``But if, in seeking to be justified in Christ, we Jews find ourselves also among the sinners, doesn't that mean that Christ promotes sin? Absolutely not!
\vs{18} If I rebuild what I destroyed, then I really would be a lawbreaker.

\vs{19} ``For through the law I died to the law so that I might live for God.
\vs{20} I have been crucified with Christ and I no longer live, but Christ lives in me. The life I now live in the body, I live by faith in the Son of God, who loved me and gave himself for me.
\vs{21} I do not set aside the grace of God, for if righteousness could be gained through the law, Christ died for nothing!''

\subsection*{\textcolor{MidnightBlue}{\textbf{Faith or Works of the Law}}}

\chapterWithIndent{3}{}


\end{document}